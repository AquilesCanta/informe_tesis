\ifx\all\undefined
\documentclass[a4paper,11pt]{book}

%----------------------------
\usepackage[utf8]{inputenc}
\usepackage[spanish]{babel}
\usepackage[T1]{fontenc}
%----------------------------
\usepackage{graphicx}
\usepackage[citecolor=black, urlcolor=black, linkcolor=black, colorlinks=true, bookmarksopen=true]{hyperref}
% \usepackage[toc]{glossaries} %xindy
% \makeglossaries
%----------------------------
\usepackage{color}
\usepackage{pdfpages}
%----------------------------
\newcommand{\sm}[0]{{\em SM}}
\newcommand{\ie}[0]{{\em i.e.,}}
\newcommand{\eg}[0]{{\em e.g.}}
\newcommand{\etal}[0]{{\em et al.}}
\newcommand{\etc}[0]{{\em etc.}}
\newcommand{\sdt}[0]{\textbf{SDT}}
\newcommand{\pat}[0]{\textbf{PAT}}
\newcommand{\cat}[0]{\textbf{CAT}}
\newcommand{\pmt}[0]{\textbf{PMT}}
\newcommand{\pid}[0]{\textbf{PID}}
\newcommand{\pes}[0]{\textbf{PES}}
\newcommand{\nit}[0]{\textbf{NIT}}





\begin{document}
\fi

\chapter{Introducci�n}

\section{Motivaci�n}


\section{Objetivos}





\subsubsection{Compatibilidad con versiones anteriores}

El est�ndar ISDB-Tb ya es utilizado en un gran n�meros de sistemas de recepci�n. Una cualidad esencial de una extensi�n con las caracter�sticas de este trabajo es la compatibilidad con versiones anteriores. Es decir, aquellos servicios que se sigan transmitiendo por radiofrecuencia, deben seguir siendo accesibles.

\subsubsection{Cambios necesarios m�nimos}

Una cualidad buscada en el dise�o debe ser la minimizaci�n del n�mero de cambios introducidos al est�ndar. Los motivos para esto son tanto mantener simplicidad en la extensi�n como requerir la menor cantidad de trabajo posible en los cambios de receptores ya funcionales.

En este cap�tulo se describe el dise�o utilizado para lograr la referencia de un \emph{ts} que viaja por radiofrecuencia y otro que viaja por un medio distinto: una red \textbf{IP}.

\subsection{En entrega de servicios IPTV}

\subsubsection{Aprovechamiento eficiente de la infraestructura}

En el caso de las redes IP, el uso eficiente de los recursos de red, como \emph{buffers} o ancho de banda es importante para proteger la infraestructura de transmisi�n, a diferencia de la emisi�n \emph{broadcast} de ISDB-Tb cuya integridad no sufre el aumento de consumidores, \eg\ es necesario enviar paquetes nulos para completar el \emph{bitrate} constante del \textbf{TS}. El env�o de paquetes nulos en el caso de las redes IP es in�til.

\subsection{En recepci�n}

\subsubsection{Conservaci�n de las interfaces}

Los cambios a realizar deber�an limitarse idealmente a la librer�a \textbf{mpeg-parser} de modo que sus clientes puedan aprovechar la extensi�n de forma transparente. Esto implicar�a que cualquiera de los reproductores de \emph{Kuntur} podr�an consumir la lista de servicios extendida.




\section{Contribuci\'on y resultados obtenidos}

\section{Estructura del documento}

\ifx\all\undefined
\end{document}
\fi
