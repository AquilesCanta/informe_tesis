\ifx\all\undefined
\documentclass[a4paper,11pt]{book}

%----------------------------
\usepackage[latin1]{inputenc}
\usepackage[spanish]{babel}
\usepackage[T1]{fontenc}
%----------------------------
\usepackage{graphicx}
\usepackage{pdfsync}
\usepackage{epsfig}
\usepackage{placeins}
\usepackage{subfig}
\usepackage{float}
%\usepackage{here}
\usepackage{listings}
\usepackage[usenames]{color}
\usepackage{courier}
\usepackage{caption}
\usepackage[ citecolor=black, urlcolor=black, linkcolor=black, colorlinks=true, bookmarksopen=true]{hyperref}
\usepackage{multirow}
\usepackage{pdfpages}
% \usepackage[toc]{glossaries} %xindy
% \makeglossaries
%----------------------------
\usepackage{color}
\definecolor{javared}{rgb}{0.6,0,0}            % strings
\definecolor{javagreen}{rgb}{0.25,0.5,0.35}    % comments
\definecolor{javapurple}{rgb}{0.5,0,0.35}      % keywords
\definecolor{javadocblue}{rgb}{0.25,0.35,0.75} % javadoc
%----------------------------
\renewcommand*{\lstlistingname}{C�digo}

\usepackage{listings}
\usepackage{courier}
\lstset{
language=[AspectJ]Java,
basicstyle=\small\ttfamily, % Standardschrift
keywordstyle=\color{javapurple}\bfseries,
stringstyle=\color{javagreen},
commentstyle=\color{javadocblue},
morecomment=[s][\color{javadocblue}]{/**}{*/},
numbers=left,
numberstyle=\tiny\color{black},
stepnumber=1,
numbersep=3pt,
tabsize=8,                  % Groesse von Tabs
extendedchars=true,         %
breaklines=true,            % Zeilen werden Umgebrochen
% keywordstyle=\color{red},
frame=b,         
%        keywordstyle=[1]\textbf,    % Stil der Keywords
%        keywordstyle=[2]\textbf,    %
%        keywordstyle=[3]\textbf,    %
%        keywordstyle=[4]\textbf,   \sqrt{\sqrt{}} %
% stringstyle=\color{white}\ttfamily, % Farbe der String
showspaces=false,           % Leerzeichen anzeigen ?
showtabs=false,             % Tabs anzeigen ?
xleftmargin=10pt,
framexleftmargin=17pt,
framexrightmargin=5pt,
framexbottommargin=4pt,
% backgroundcolor=\color{lightgray},
showstringspaces=false      % Leerzeichen in Strings anzeigen ?        
}

\usepackage{caption}
\DeclareCaptionFont{white}{\color{white}}
\DeclareCaptionFormat{listing}{\colorbox[cmyk]{0.43, 0.35, 0.35,0.10}{\parbox{\textwidth}{\hspace{15pt}#1#2#3}}}
\captionsetup[lstlisting]{format=listing,labelfont=white,textfont=white, singlelinecheck=false, margin=0pt, font={bf,footnotesize}}

%----------------------------
\usepackage{anysize}
\marginsize{4cm}{3cm}{2cm}{2cm}
%----------------------------
\newcommand{\toref}[1]{\textbf{[REFERENCE: #1]}}
\newcommand{\tocomplete}[1]{\textbf{[Complete: #1 ]}}
\newcommand{\todelete}[1]{\textbf{[Delete this: \textcolor{cyan}{#1}]}}
\newcommand{\tocheck}[1]{\textbf{[Check this!: #1 ]}}
\newcommand{\todo}[1]{\textbf{[To Do:{#1}]}}
\newcommand{\edited}[1]{\textcolor{red}{#1}}
\newcommand{\comment}[1]{\textcolor{blue}{#1}}
\newcommand{\newstuff}[1]{\textcolor{green}{#1}}
\newcommand{\mrk}[0]{}

%----------------------------
% espacio entre parrafos
\setlength{\parskip}{5mm}
% sangria
\setlength{\parindent}{5mm}

\linespread{1.1}%
\selectfont	
% correct bad hyphenation here
\hyphenation{ in-te-rac-ci�n conexio-nes }

%-----------------------------
\newcommand{\sm}[0]{{\em SM}}
\newcommand{\ie}[0]{{\em i.e.,}}
\newcommand{\eg}[0]{{\em e.g.}}
\newcommand{\etal}[0]{{\em et al.}}
\newcommand{\etc}[0]{{\em etc.}}

\newcommand{\cn}[0]{{\textit{concern}}}
\newcommand{\cns}[0]{{\textit{concerns}}}
\newcommand{\ccc}[0]{{\textit{crosscutting concerns}}}


\newcommand{\game}[0]{{\emph{Game}}}
\newcommand{\meters}[0]{{\emph{Meters}}}
\newcommand{\recall}[0]{{\emph{Game Recall}}}
\newcommand{\error}[0]{{\emph{Error Condition}}}
\newcommand{\errors}[0]{{\emph{Errors Conditions}}}
\newcommand{\comm}[0]{{\emph{Communication Protocol}}}
\newcommand{\comms}[0]{{\emph{Communication Protocols}}}
\newcommand{\demo}[0]{{\emph{Demo}}}
\newcommand{\resump}[0]{{\emph{Program Resumption}}}
\begin{document}
\fi

\chapter{Modelo de Objetos}
En el cap\'itulo 4 se decriben detalles de implementaci\'on de cada \cn~teniendo en cuenta como los objetos interact\'uan con los aspectos del sistema.
A continuaci\'on se detallan las reponsabilidades de los objetos involucrados en la soluci\'on.

\section{Paquete Game}
 
Las clases del paquete \textit{Game} se pueden ver en la figura \ref{fig:game_classes}.

\begin{figure}[ht!]
\centering
\includegraphics[scale=0.60]{img/modelo/game.eps}
\caption{Clases del paquete \textit{Game}.}
\label{fig:game_classes}
\end{figure}

\clearpage

\begin{itemize}
\item \texttt{Game}
\begin{itemize}
\item implementa la interfaz de alto nivel de un juego de apuestas.
\item el juego puede estar habilitado o deshabilitado.
\item un juego necesita de cr�ditos para realizar una jugada, adem\'as de estar habilitado.
\item en cada jugada se descuentan los cr\'editos que se apuestan y se suman los ganados, si es que los hubo. 
\item se puede jugar mientras queden cr\'editos o hasta que el jugador retira los mismos.
\newline
\end{itemize}

\item \texttt{BetManager}
\begin{itemize}
\item Maneja el sistema de apuestas del juego: configuraci�n de las l�neas de pago, cantidad de l�neas apostadas, cantidad de cr\'editos apostados por l�nea.
\item Conoce el costo de cada jugada ( \textit{betPerLine} x \textit{selectedPaylines} ).
\item Define una colecci�n de objetos \texttt{Payline}, los cuales se utilizan para saber si hay l�neas ganadoras entre las seleccionadas.
\newline
\end{itemize}

\item \texttt{Payline}: 
\begin{itemize}
\item Representa una l\'inea de apuesta en un juego de \textit{reels}. 
\item Una \textit{payline} puede estar activa, es decir apostada o no.
\item Permite determinar que cantidad de s\'imbolos iguales consecutivos hay en sus posiciones y con esto saber si es o no una l�nea ganadora.
\newline
\end{itemize}

\item \texttt{ReelsManager}: 
\begin{itemize}
\item Crea e inicializa 5 objetos de la clase \texttt{Reel}.
\item La operaci\'on de \texttt{spin()} delega en cada \textit{reel} el sorteo de la nueva posici�n.
\item Permite consultar la ventana de s\'imbolos resultante del sorteo.
\newline
\end{itemize}

\item \texttt{Reel}: 
\begin{itemize}
\item Modela una tira circular de s\'imbolos.
\item El m\'etodo \texttt{spin()} genera una nuevo \textit{offset} de manera aleatoria.
\end{itemize}

\end{itemize}


\clearpage
\section{Paquete HAL}
Las clases del paquete \textit{HAL} se pueden ver en la figura \ref{fig:hal_classes}.


\begin{figure}[ht!]
\centering
\includegraphics[scale=0.60]{img/modelo/hal.eps}
\caption{Clases del paquete \textit{HAL}.}
\label{fig:hal_classes}
\end{figure}
\FloatBarrier
\begin{itemize}

\item \texttt{Device}
\begin{itemize}
\item Representa un disposivo de \textit{hardware}.
\item Un disposivo puede estar o no activo.
\newline
\end{itemize}

\item \texttt{CoinAcceptor}
\begin{itemize}
\item Esta clase representa un disposivo que permite el ingreso de monedas al juego a trav\'es de la \textit{HAL}.
\item Una moneda ingresada puede:
\begin{itemize}
\item ser v\'alida y tener un valor reconocido.
\item ser inv\'alida (\textit{invalidCoin}).
\item trabar y deshabilitar el dispositivo (\textit{coinTilt}).
\newline
\end{itemize}
\end{itemize}

\item \texttt{Door}
\begin{itemize}
\item Esta clase representa un puerta del gabinete, como puede ser la \textit{main door}.
\item Una EGM tiene un puerta principal que de ser abierta, debe deshabilitar el juego por completo.
\newline
\end{itemize}

\item \texttt{NVRAM}
\begin{itemize}
\item Esta clase representa un dispositivo de memoria de acceso aleatorio no vol�til.
\item Permite guardar datos como un par ( clave, valor ).
\item Permite recuperar datos a partir de una clave.
\newline
\end{itemize}

\item \texttt{TowerLamp}
\begin{itemize}
\item Representa la l\'ampara que tiene una EGM que puede encenderse o apagarse de acuerdo a los distintos estados de la EGM.
\newline
\end{itemize}

\item \texttt{HAL}
\begin{itemize}
\item \textit{Hardware Abstraction Layer}
\item Esta clase abstrae de la comunicaci\'on con los dispositivos f\'isicos al resto de los componentes del \textit{software}.
\newline
\end{itemize}

\end{itemize}

\section{Paquete Meters}
 
Las clases del paquete \textit{Meters} se pueden ver en la figura \ref{fig:meters_classes}.

\begin{figure}[ht!]
\centering
\includegraphics[scale=0.60]{img/modelo/meters.eps}
\caption{Clases del paquete \textit{Meters}.}
\label{fig:meters_classes}
\end{figure}

\begin{itemize}
\item \texttt{Meter}
\begin{itemize}
\item Esta clase se utiliza para definir un tipo enumerativo. El mismo es usado para el manejo de \textit{meters}.
\item Algunos valores son: GAMES, CREDITS, LINES, WIN, CASHOUT.
\newline
\end{itemize}

\item \texttt{MetersManager}
\begin{itemize}
\item Mantiena un mapa de \textit{meters}.
\item Para el acceso a los mismos se utiliza el tipo enumerativo anteriormente descripto.
\item Para un \textit{meter} en particular esta clase permite: 
\begin{itemize}
\item asignar un valor espec\'ifico.
\item consultar el valor actual.
\item incrementar el valor actual.
\item decrementar el valor actual.
\end{itemize}
\end{itemize}

\end{itemize}

\section{Paquete Game Recall}
 
Las clases del paquete \textit{Game Recall} se pueden ver en la figura \ref{fig:gameRecall_classes}.

\begin{figure}[ht!]
\centering
\includegraphics[scale=0.60]{img/modelo/gameRecall.eps}
\caption{Clases del paquete \textit{Game Recall}.}
\label{fig:gameRecall_classes}
\end{figure}

\begin{itemize}
\item \texttt{GameRecall}
\begin{itemize}
\item Un objeto \texttt{GameRecall} mantiene informaci�n de una jugada en particular.
\item Esta informaci�n se compone de:
\begin{itemize}
\item fecha y hora en que se realiz\'o la jugada.
\item datos de la apuesta realizada (\textit{bet}, \textit{lines}).
\item cantidad de cr\'editos que hab\'ia en el juego antes de realizar la jugada.
\item cantidad de cr\'editos ganados.
\item s\'imbolos de los \textit{reels} que resultaron del sorteo.
\newline
\end{itemize}

\end{itemize}

\item \texttt{GameRecallManager}
\begin{itemize}
\item Mantiene una pila con las \'ultimos \textit{game recalls}.
\item Permite agregar un nuevo objeto \texttt{GameRecall}.
\item Permite consultar el contenido de la pila.
\end{itemize}

\end{itemize}

\section{Paquete Errors}

Las clases del paquete \textit{Errors} se pueden ver en la figura \ref{fig:errors_classes}.

\begin{figure}[ht!]
\centering
\includegraphics[scale=0.60]{img/modelo/errors.eps}
\caption{Clases del paquete \textit{Erros}.}
\label{fig:errors_classes}
\end{figure}


\begin{itemize}
\item \texttt{Action}
\begin{itemize}
\item Representa una acci�n a ser realizada ante una \textit{error condition}.
\item Una acci�n puede aplicarse o desaplicarse.
\item Las clases \texttt{DisableGame}, \texttt{TurnOffInputDevices} y \texttt{TowerLampOn} implementan la interfaz \texttt{Action}.
\end{itemize}

\item \texttt{ErrorCondition}
\begin{itemize}
\item Es una clase abstracta que contiene las acciones que se toman al generarse una \textit{error condition} particular.
\item Al generarse una \textit{error condition} se aplican las acciones.
\item Cuando una \textit{error condition} deja de estar presente, se desaplican las acciones.
\end{itemize}

\item \texttt{DoorOpen}
\begin{itemize}
 \item Es un tipo de \texttt{ErrorCondition} que se genera al abrirse un puerta y se cancela al cerrarse la misma.
 \newline
\end{itemize}
\end{itemize}

\clearpage
\section{Paquete Protocols}

Las clases del paquete \textit{Protocols} se pueden ver en la figura \ref{fig:protocols_classes}.

\begin{figure}[ht!]
\centering
\includegraphics[scale=0.60]{img/modelo/protocols.eps}
\caption{Clases del paquete \textit{Protocols}.}
\label{fig:protocols_classes}
\end{figure}

\begin{itemize}
\item \texttt{ProtocolCmd}
\begin{itemize}
 \item Es una interfaz que deben implementar los comandos que se env\'ian y reciben a trav\'es de un protocolo de comunicaci�n.
 \item Las siguientes clases implementan dicha interfaz:
 \begin{itemize}
 \item \texttt{SetTime} este comando al ejecutarse configura la hora de la EGM.
 %\item \textit{SetProgressive}
 \item \texttt{GetMeter} al recibir este comando debe retornarse el valor del \textit{meter} requerido.
 \item \texttt{GetGameRecall} al recibir este comando se retorna la lista de \textit{game recalls}.
 \newline	
\end{itemize}
\end{itemize}

\item \texttt{CommProtocol}
\begin{itemize}
 \item Una instancia de esta clase, es un \textit{thread} que esta a la espera de recibir un objeto \texttt{ProtocolCmd}.
 \item Tambi\'en tiene la responsabilidad de enviar a los protocolos los eventos que ocurren en la \sm.
 \item Al recibirse un \textit{ProtocolCmd} se ejecuta la acci�n asociada al comando.
\end{itemize}
\end{itemize}

\ifx\all\undefined
\end{document}
\fi