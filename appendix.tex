\ifx\all\undefined
\documentclass[a4paper,11pt]{book}

%----------------------------
\usepackage[utf8]{inputenc}
\usepackage[spanish]{babel}
\usepackage[T1]{fontenc}
%----------------------------
\usepackage{graphicx}
\usepackage[citecolor=black, urlcolor=black, linkcolor=black, colorlinks=true, bookmarksopen=true]{hyperref}
% \usepackage[toc]{glossaries} %xindy
% \makeglossaries
%----------------------------
\usepackage{color}
\usepackage{pdfpages}
%----------------------------
\newcommand{\sm}[0]{{\em SM}}
\newcommand{\ie}[0]{{\em i.e.,}}
\newcommand{\eg}[0]{{\em e.g.}}
\newcommand{\etal}[0]{{\em et al.}}
\newcommand{\etc}[0]{{\em etc.}}
\newcommand{\sdt}[0]{\textbf{SDT}}
\newcommand{\pat}[0]{\textbf{PAT}}
\newcommand{\cat}[0]{\textbf{CAT}}
\newcommand{\pmt}[0]{\textbf{PMT}}
\newcommand{\pid}[0]{\textbf{PID}}
\newcommand{\pes}[0]{\textbf{PES}}
\newcommand{\nit}[0]{\textbf{NIT}}





\begin{document}
\fi

\newpage

\chapter{Merengue merengue} \label{App:AppendixA}

Saracatunga

\begin{figure}[h!]
	\includegraphics[width=12cm]{imgs/mpeg_structure.png}
	\caption{Diagrama de composici�n de un flujo de transporte} 
	\centering 
\end{figure}

\begin{itemize}
\item \textbf{Red de transmisi�n o \emph{network}:} Uno o m�s flujos de transporte emitidos por una misma entidad.
\item \textbf{Flujo de transporte:} Un flujo \textbf{MPEG-2} que lleva uno o m�s servicios.
\item \textbf{Servicio:} Alg�n contenido que puede consumir un receptor, \eg\ un canal de televisi�n, un canal de radio o un servicio de ingenier�a. 
\item \textbf{Evento:} Un programa de televisi�n. Dos eventos de un mismo servicio s�lo est�n diferenciados por el momento en que se est� sintonizando un servicio espec�fico.
\item \textbf{Flujo elemental o \emph{elementary stream}:} Un evento est� compuesto por uno o m�s flujos elementales. A su vez, puede contener m�s de un flujo del mismo tipo. \eg\ un show de televisi�n puede estar emiti�ndose en varios idiomas al mismo tiempo (un audio en ingl�s y otro en espa�ol). Los flujos elementales requieren de la mayor parte del ancho de banda de un flujo de transporte.
\end{itemize}

\subsubsection{Ejemplo de transport stream}

Continuando con el ejemplo introducido previamente transmitido por la frecuencia 527 Mhz, se puede ahondar en el an�lisis a trav�s de la jerarqu�a anterior:

\begin{itemize}
\item \textbf{Red de transmisi�n o \emph{network}:} El nombre de la red de transmisi�n es ``RTA C23'' que es administrada por Radio y Televisi�n Argentina S.E[TODO: Agregar cita]. 
\item \textbf{Flujo de transporte:} El flujo de transporte son los contenidos enviados por el canal f�sico 23 correspondiente a la frecuencia 527 Mhz.
\item \textbf{Servicios:} Presenta 3. TV P�blica HD, Tecn�polis y TV P�blica.
\item \textbf{Evento:} Son los distintos shows de cada canal. Por ejemplo, a las 14:00 horas del 5 de Junio comienza una emisi�n de ``Ciencia ahora''.
\item \textbf{Flujo elemental o \emph{elementary stream}:} Los flujos elementales pueden cambiar dependiendo del evento emitido en el momento. En el caso del fragmento utilizado, presenta 4 audios y 3 videos. Existen en el flujo de transporte otros \emph{elementary streams} que caen fuera del alcance de este informe.
\end{itemize}

\newpage
\chapter{Contribuci\'on y resultados obtenidos}

\newpage
\chapter{Temas de ap�ndice}

\begin{itemize}
\item Estructura del paquete de transport stream / Repasar el formato mpeg entero

\item Lista completa de las tablas de ISDB-Tb. Juntar con la de abajo.
\item Lista completa de los descriptores ISDB-Tb. Juntar con el de abajo
\item Lista de pids reservados, puede ir junto a lo de la tabla.

\item Comunicaci�n de threads en Wari
\item Descompresi�n de audio y video? 

\item Compilaci�n de Mara?
\item Construcci�n y utilizaci�n de herramientas en wari
\item OpenCaster?
\item DSM-CC explicado quiz� es demasiado.
\end{itemize}

\ifx\all\undefined
\end{document}
\fi



%%%%%%%%%%%%%%%%%%%%%%%%%%%%%%%%%%%%%%%%%%%%%%%%%%%%%%%%%%%%%%%%%%%%%%%%%%%%%%%%%%%%%%%%%%%%%%%%%%%%%%%%%%%%%%%%%%%%%
