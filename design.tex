\ifx\all\undefined
\documentclass[a4paper,11pt]{book}

%----------------------------
\usepackage[latin1]{inputenc}
\usepackage[spanish]{babel}
\usepackage[T1]{fontenc}
%----------------------------
\usepackage{graphicx}
\usepackage{pdfsync}
\usepackage{epsfig}
\usepackage{placeins}
\usepackage{subfig}
\usepackage{float}
%\usepackage{here}
\usepackage{listings}
\usepackage[usenames]{color}
\usepackage{courier}
\usepackage{caption}
\usepackage[ citecolor=black, urlcolor=black, linkcolor=black, colorlinks=true, bookmarksopen=true]{hyperref}
\usepackage{multirow}
\usepackage{pdfpages}
% \usepackage[toc]{glossaries} %xindy
% \makeglossaries
%----------------------------
\usepackage{color}
\definecolor{javared}{rgb}{0.6,0,0}            % strings
\definecolor{javagreen}{rgb}{0.25,0.5,0.35}    % comments
\definecolor{javapurple}{rgb}{0.5,0,0.35}      % keywords
\definecolor{javadocblue}{rgb}{0.25,0.35,0.75} % javadoc
%----------------------------
\renewcommand*{\lstlistingname}{C�digo}

\usepackage{listings}
\usepackage{courier}
\lstset{
language=[AspectJ]Java,
basicstyle=\small\ttfamily, % Standardschrift
keywordstyle=\color{javapurple}\bfseries,
stringstyle=\color{javagreen},
commentstyle=\color{javadocblue},
morecomment=[s][\color{javadocblue}]{/**}{*/},
numbers=left,
numberstyle=\tiny\color{black},
stepnumber=1,
numbersep=3pt,
tabsize=8,                  % Groesse von Tabs
extendedchars=true,         %
breaklines=true,            % Zeilen werden Umgebrochen
% keywordstyle=\color{red},
frame=b,         
%        keywordstyle=[1]\textbf,    % Stil der Keywords
%        keywordstyle=[2]\textbf,    %
%        keywordstyle=[3]\textbf,    %
%        keywordstyle=[4]\textbf,   \sqrt{\sqrt{}} %
% stringstyle=\color{white}\ttfamily, % Farbe der String
showspaces=false,           % Leerzeichen anzeigen ?
showtabs=false,             % Tabs anzeigen ?
xleftmargin=10pt,
framexleftmargin=17pt,
framexrightmargin=5pt,
framexbottommargin=4pt,
% backgroundcolor=\color{lightgray},
showstringspaces=false      % Leerzeichen in Strings anzeigen ?        
}

\usepackage{caption}
\DeclareCaptionFont{white}{\color{white}}
\DeclareCaptionFormat{listing}{\colorbox[cmyk]{0.43, 0.35, 0.35,0.10}{\parbox{\textwidth}{\hspace{15pt}#1#2#3}}}
\captionsetup[lstlisting]{format=listing,labelfont=white,textfont=white, singlelinecheck=false, margin=0pt, font={bf,footnotesize}}

%----------------------------
\usepackage{anysize}
\marginsize{4cm}{3cm}{2cm}{2cm}
%----------------------------
\newcommand{\toref}[1]{\textbf{[REFERENCE: #1]}}
\newcommand{\tocomplete}[1]{\textbf{[Complete: #1 ]}}
\newcommand{\todelete}[1]{\textbf{[Delete this: \textcolor{cyan}{#1}]}}
\newcommand{\tocheck}[1]{\textbf{[Check this!: #1 ]}}
\newcommand{\todo}[1]{\textbf{[To Do:{#1}]}}
\newcommand{\edited}[1]{\textcolor{red}{#1}}
\newcommand{\comment}[1]{\textcolor{blue}{#1}}
\newcommand{\newstuff}[1]{\textcolor{green}{#1}}
\newcommand{\mrk}[0]{}

%----------------------------
% espacio entre parrafos
\setlength{\parskip}{5mm}
% sangria
\setlength{\parindent}{5mm}

\linespread{1.1}%
\selectfont	
% correct bad hyphenation here
\hyphenation{ in-te-rac-ci�n conexio-nes }

%-----------------------------
\newcommand{\sm}[0]{{\em SM}}
\newcommand{\ie}[0]{{\em i.e.,}}
\newcommand{\eg}[0]{{\em e.g.}}
\newcommand{\etal}[0]{{\em et al.}}
\newcommand{\etc}[0]{{\em etc.}}

\newcommand{\cn}[0]{{\textit{concern}}}
\newcommand{\cns}[0]{{\textit{concerns}}}
\newcommand{\ccc}[0]{{\textit{crosscutting concerns}}}


\newcommand{\game}[0]{{\emph{Game}}}
\newcommand{\meters}[0]{{\emph{Meters}}}
\newcommand{\recall}[0]{{\emph{Game Recall}}}
\newcommand{\error}[0]{{\emph{Error Condition}}}
\newcommand{\errors}[0]{{\emph{Errors Conditions}}}
\newcommand{\comm}[0]{{\emph{Communication Protocol}}}
\newcommand{\comms}[0]{{\emph{Communication Protocols}}}
\newcommand{\demo}[0]{{\emph{Demo}}}
\newcommand{\resump}[0]{{\emph{Program Resumption}}}
\begin{document}
\fi
 
\chapter{Dise�o}

El objetivo de este trabajo es el desarrollo de un prototipo de sistema que integre transmisi�n ISDB-Tb e IPTV. Este cap�tulo trata las decisiones tomadas m�s importantes en el proceso.

Hay tres principales v�as de trabajo.

\begin{itemize}
\item Entrega de servicios por ISDB-Tb
\item Entrega de servicios por redes IP (o IPTV)
\item Recepci�n y combinaci�n de ambas
\end{itemize}

\section{Principios de dise�o}

\subsection{En entrega de servicios ISDB-Tb}

\subsubsection{Compatibilidad con versiones anteriores}

El est�ndar ISDB-Tb ya es utilizado en un gran n�meros de sistemas de recepci�n. Una cualidad esencial de una extensi�n con las caracter�sticas de este trabajo es la compatibilidad con versiones anteriores. Es decir, aquellos servicios que se sigan transmitiendo por radiofrecuencia, deben seguir siendo accesibles.

\subsubsection{Cambios necesarios m�nimos}

Una cualidad buscada en el dise�o debe ser la minimizaci�n del n�mero de cambios introducidos al est�ndar. Los motivos para esto son tanto mantener simplicidad en la extensi�n como requerir la menor cantidad de trabajo posible en los cambios de receptores ya funcionales.

En este cap�tulo se describe el dise�o utilizado para lograr la referencia de un \emph{ts} que viaja por radiofrecuencia y otro que viaja por un medio distinto: una red \textbf{IP}.

\subsection{En entrega de servicios IPTV}

\subsubsection{Aprovechamiento eficiente de la infraestructura}

En el caso de las redes IP, el uso eficiente de los recursos de red, como \emph{buffers} o ancho de banda es importante para proteger la infraestructura de transmisi�n, a diferencia de la emisi�n \emph{broadcast} de ISDB-Tb cuya integridad no sufre el aumento de consumidores, \eg\ es necesario enviar paquetes nulos para completar el \emph{bitrate} constante del \textbf{TS}. El env�o de paquetes nulos en el caso de las redes IP es in�til.

\subsection{Televisi�n en vivo}

La transmisi�n \emph{broadcast}, como es el caso de ISDB-Tb, clasifica naturalmente en la primera categor�a. 


\section{Extensi�n de la lista de servicios}

Se debe agregar una entrada a la PAT
Se debe agregar un descriptor a la SDT (OJO! Puede implicar la necesidad de dividir la secci�n en dos paquetes!)
Se debe crear una nueva PMT e insertarla en el TS v�lido de partida.

\section{Realocaci�n de los \emph{elementary streams} de los servicios agregados}

\ifx\all\undefined
\end{document}
\fi