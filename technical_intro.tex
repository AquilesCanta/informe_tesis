\ifx\all\undefined
\documentclass[spanish,a4paper,11pt]{book}

%----------------------------
\usepackage[spanish,es-noquoting,es-tabla]{babel}
\usepackage[latin1]{inputenc}
\usepackage[T1]{fontenc}
%----------------------------
\usepackage{graphicx}
\usepackage[citecolor=black, urlcolor=black, linkcolor=black, colorlinks=true, bookmarksopen=true]{hyperref}
\usepackage{cleveref}
\usepackage{float}
\usepackage{enumitem}
% \usepackage[toc]{glossaries} %xindy
% \makeglossaries
%----------------------------
\usepackage{amsmath}

\makeatletter
\newcommand*{\bdiv}{%
  \nonscript\mskip-\medmuskip\mkern5mu%
  \mathbin{\operator@font div}\penalty900\mkern5mu%
  \nonscript\mskip-\medmuskip
}
\makeatother

\usepackage[]{geometry}
\usepackage{tikz}
\usepackage[simplified]{pgf-umlcd}
\usepackage{pgf-umlsd}
%----------------------------
%CODE:
\usepackage[procnames]{listings}
\usepackage{color}
\usepackage{tablefootnote}

\definecolor{keywords}{RGB}{255,0,90}
\definecolor{comments}{RGB}{0,0,113}
\definecolor{red}{RGB}{160,0,0}
\definecolor{green}{RGB}{0,90,0}
\definecolor{gray}{RGB}{190,190,190}
\definecolor{mauve}{RGB}{160,126,175}

\crefname{table}{\spanishtablename}{\spanishtablename}
\crefname{listing}{código fuente}{códigos fuente}

\lstset{language=Python, 
        basicstyle=\ttfamily\small, 
        keywordstyle=\color{keywords},
        commentstyle=\color{comments},
        stringstyle=\color{red},
        breaklines=true, 
        showstringspaces=false,
        identifierstyle=\color{green},
        tabsize=2,
        procnamekeys={def,class}
        numbers=left,
        frame=single,
        numbers=left,
        numbersep=5pt,
        numberstyle=\tiny
}

% \lstset{language=C++,
% basicstyle=\footnotesize\sffamily\color{black},
% commentstyle=\color{gray},
% frame=single,
% numbers=left,
% numbersep=5pt,
% numberstyle=\tiny\color{gray},
% keywordstyle=\color{green},
% showspaces=false,
% showstringspaces=false,
% stringstyle=\color{orange},
% tabsize=2
% }

% \lstset{ %
%   backgroundcolor=\color{white},   % choose the background color; you must add \usepackage{color} or \usepackage{xcolor}
%   basicstyle=\footnotesize,        % the size of the fonts that are used for the code
%   breakatwhitespace=false,         % sets if automatic breaks should only happen at whitespace
%   breaklines=true,                 % sets automatic line breaking
%   captionpos=b,                    % sets the caption-position to bottom
%   commentstyle=\color{green},    % comment style
%   %deletekeywords={...},            % if you want to delete keywords from the given language
%   escapeinside={\%*}{*)},          % if you want to add LaTeX within your code
%   extendedchars=true,              % lets you use non-ASCII characters; for 8-bits encodings only, does not work with UTF-8
%   frame=single,                    % adds a frame around the code
%   keepspaces=true,                 % keeps spaces in text, useful for keeping indentation of code (possibly needs columns=flexible)
%   keywordstyle=\color{blue},       % keyword style
%   language=Octave,                 % the language of the code
%   morekeywords={},            % if you want to add more keywords to the set
%   numbers=left,                    % where to put the line-numbers; possible values are (none, left, right)
%   numbersep=5pt,                   % how far the line-numbers are from the code
%   numberstyle=\tiny\color{gray}, % the style that is used for the line-numbers
%   rulecolor=\color{black},         % if not set, the frame-color may be changed on line-breaks within not-black text (e.g. comments (green here))
%   showspaces=false,                % show spaces everywhere adding particular underscores; it overrides 'showstringspaces'
%   showstringspaces=false,          % underline spaces within strings only
%   showtabs=false,                  % show tabs within strings adding particular underscores
%   stepnumber=2,                    % the step between two line-numbers. If it's 1, each line will be numbered
%   stringstyle=\color{mauve},     % string literal style
%   tabsize=2,                       % sets default tabsize to 2 spaces
%   title=\lstname                   % show the filename of files included with \lstinputlisting; also try caption instead of title
% }

\crefname{lstlisting}{fragmento}{fragmentos}
\Crefname{lstlisting}{Fragmento}{Fragmentos}
\renewcommand\lstlistingname{Fragmento}
\renewcommand\lstlistlistingname{Fragmentos}
\def\lstlistingautorefname{Frag.}

\usepackage{pdfpages}
%----------------------------
\newcommand{\ie}[0]{{\em i.e.,}}
\newcommand{\eg}[0]{{\em e.g.}}
\newcommand{\etal}[0]{{\em et al.}}
\newcommand{\etc}[0]{{\em etc.}}
\newcommand{\sdt}[0]{\textbf{SDT}}
\newcommand{\pat}[0]{\textbf{PAT}}
\newcommand{\cat}[0]{\textbf{CAT}}
\newcommand{\pmt}[0]{\textbf{PMT}}
\newcommand{\pid}[0]{\textbf{PID}}
\newcommand{\pes}[0]{\textbf{PES}}
\newcommand{\nit}[0]{\textbf{NIT}}


\renewenvironment{package}[2][\umlcdPackageTitle]{
\edef\umlcdPackageTitle{#2}
\def\umlcdPackageFit{}
\def\umlcdPackageName{#1}
}{
  \begin{pgfonlayer}{background}
  \node[umlcd style, draw, inner sep=0.5cm, fit = \umlcdPackageFit] (\umlcdPackageName) {};
  \node[umlcd style, draw, outer ysep=-0.5, anchor=south west] (\umlcdPackageName caption) at
  (\umlcdPackageName.north west) {\umlcdPackageTitle};
  \end{pgfonlayer}
}
\begin{document}
\fi
\chapter{Introducci�n t�cnica}

\textbf{ISDB-Tb} es un est�ndar de transmisi�n de televisi�n digital terrestre de origen japon�s con modificaciones brasile�as. Es el adoptado en Argentina. La multiplexaci�n de los servicios respeta el est�ndar \textbf{MPEG-2 Transport Stream}. Los reproductores de televisi�n digital terrestre deben utilizar estas normas para capturar, demultiplexar, procesar y entregar los servicios al usuario. \textbf{Lifia} lleva adelante el desarrollo del proyecto \emph{Kuntur} que incluye, entre otras cosas, el reproductor \emph{Wari}.

\section{Transmisi�n}

La televisi�n terrestre aprovecha el espectro de radiofrecuencia para la transmisi�n de contenido. La radiofrecuencia se divide en canales f�sicos de $6 MHz$ por los que viaja un flujo multiplexado de servicios. Cada uno de estos canales dispone de un \emph{bitrate} constante definido por los par�metros de modulaci�n.

Esta limitaci�n de bitrate establece una cota superior para el n�mero se servicios ofrecidos. Dependiendo ulteriormente del ancho de banda requerido por los mismos.

\section{Multiplexaci�n}

El formato del flujo multiplexado de servicios est� dado por el formato est�ndar \textbf{MPEG-2 TS}. El mismo se compone de paquetes de tama�o fijo de 188 \emph{bytes} que contienen un prefijo de 4 \emph{bytes} con varias utilidades, como identificar el contenido del mismo.

Los contenidos de los paquetes de \textbf{Transport stream} llevan partes de \emph{secciones} o \emph{packetized elementary streams}.

\section{Decodificaci�n}

El reproductor \emph{Wari} se utiliza la librer�a \textbf{mpeg-parser}, tambi�n incluida en el proyecto \emph{Kuntur}, para la manipulaci�n de \textbf{TS}. La responsabilidad de esta liber�a es capturar los paquetes de \textbf{Transport stream}, reconstruir secciones y flujos elementales, para identificar los servicios y poder entregarlos para su reproducci�n.

El formato \emph{transport stream} se divide en dos categor�as: \emph{sistema} y \emph{compresi�n}. La primera capa corresponde a la informaci�n espec�fica de multiplexaci�n de los servicios. Se�aliza los contenidos del flujo de transporte para su posterior demultiplexaci�n. Tambi�n incluye informaci�n espec�fica de la red (\emph{network}) a la que pertenece el \textbf{Transport Stream}. La principal componente de la capa de sistema son las tablas que son transportadas por secciones del \emph{ts}.

\subsection{Secciones y tablas}

El formato \textbf{MPEG-TS} define varias tablas. Puede definirse cierta jerarqu�a sobre las mismas.

\subsubsection{\emph{Program allocation table}}

La \textbf{PAT} entrega al decodificador la asociaci�n \textbf{PMT}(que se explica a continuaci�n) con \textbf{PID}. Cada \textbf{PMT} corresponde a un servicio incluido en el \textbf{Transport stream}. A cada uno de estos le corresponde un n�mero identificador que ser� utilizado por el resto de las tablas para adjuntar informaci�n espec�fica, como el nombre o su tipo.

\subsubsection{\emph{Program mapping table}}

Lorore rore

\subsubsection{\emph{Service description table}}

SDT++

\subsection{\emph{Packetized elementary streams}}

El audio y video enviados en un \emph{ts} utilizan paquetes \textbf{PES}, que poseen longitud variable. Estos paquetes se dividen luego entre paquetes \emph{ts} para ser multiplexados luegos al flujo de transporte final.

\ifx\all\undefined
\end{document}
\fi