\ifx\all\undefined
\documentclass[a4paper,11pt]{book}

%----------------------------
\usepackage[utf8]{inputenc}
\usepackage[spanish]{babel}
\usepackage[T1]{fontenc}
%----------------------------
\usepackage{graphicx}
\usepackage[citecolor=black, urlcolor=black, linkcolor=black, colorlinks=true, bookmarksopen=true]{hyperref}
% \usepackage[toc]{glossaries} %xindy
% \makeglossaries
%----------------------------
\usepackage{color}
\usepackage{pdfpages}
%----------------------------
\newcommand{\sm}[0]{{\em SM}}
\newcommand{\ie}[0]{{\em i.e.,}}
\newcommand{\eg}[0]{{\em e.g.}}
\newcommand{\etal}[0]{{\em et al.}}
\newcommand{\etc}[0]{{\em etc.}}
\newcommand{\sdt}[0]{\textbf{SDT}}
\newcommand{\pat}[0]{\textbf{PAT}}
\newcommand{\cat}[0]{\textbf{CAT}}
\newcommand{\pmt}[0]{\textbf{PMT}}
\newcommand{\pid}[0]{\textbf{PID}}
\newcommand{\pes}[0]{\textbf{PES}}
\newcommand{\nit}[0]{\textbf{NIT}}





\begin{document}
\fi
\chapter{Introducci�n t�cnica}

\textbf{ISDB-Tb} es un est�ndar de transmisi�n de televisi�n digital terrestre de origen japon�s con modificaciones brasile�as. Es el adoptado en Argentina. La multiplexaci�n de los servicios respeta el est�ndar \textbf{MPEG-2 Transport Stream}. Los reproductores de televisi�n digital terrestre deben utilizar estas normas para capturar, demultiplexar, procesar y entregar los servicios al usuario. Lifia lleva adelante el desarrollo del proyecto \emph{Kuntur} que incluye, entre otras cosas, el reproductor \emph{Wari}.

\section{Transmisi�n}

La televisi�n terrestre aprovecha el espectro de radiofrecuencia para la transmisi�n de contenido. La radiofrecuencia se divide en canales f�sicos de $6 MHz$ por los que viaja un flujo multiplexado de servicios. Cada uno de estos canales dispone de un \emph{bitrate} constante definido por los par�metros de modulaci�n.

Esta limitaci�n de bitrate establece una cota superior para el n�mero se servicios ofrecidos. Dependiendo ulteriormente del ancho de banda requerido por los mismos.

\section{Multiplexaci�n}

\section{Decodificaci�n}

\ifx\all\undefined
\end{document}
\fi