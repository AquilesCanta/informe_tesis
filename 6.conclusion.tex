\ifx\all\undefined
\documentclass[a4paper,11pt]{book}

%----------------------------
\usepackage[latin1]{inputenc}
\usepackage[spanish]{babel}
\usepackage[T1]{fontenc}
%----------------------------
\usepackage{graphicx}
\usepackage{pdfsync}
\usepackage{epsfig}
\usepackage{placeins}
\usepackage{subfig}
\usepackage{float}
%\usepackage{here}
\usepackage{listings}
\usepackage[usenames]{color}
\usepackage{courier}
\usepackage{caption}
\usepackage[ citecolor=black, urlcolor=black, linkcolor=black, colorlinks=true, bookmarksopen=true]{hyperref}
\usepackage{multirow}
\usepackage{pdfpages}
% \usepackage[toc]{glossaries} %xindy
% \makeglossaries
%----------------------------
\usepackage{color}
\definecolor{javared}{rgb}{0.6,0,0}            % strings
\definecolor{javagreen}{rgb}{0.25,0.5,0.35}    % comments
\definecolor{javapurple}{rgb}{0.5,0,0.35}      % keywords
\definecolor{javadocblue}{rgb}{0.25,0.35,0.75} % javadoc
%----------------------------
\renewcommand*{\lstlistingname}{C�digo}

\usepackage{listings}
\usepackage{courier}
\lstset{
language=[AspectJ]Java,
basicstyle=\small\ttfamily, % Standardschrift
keywordstyle=\color{javapurple}\bfseries,
stringstyle=\color{javagreen},
commentstyle=\color{javadocblue},
morecomment=[s][\color{javadocblue}]{/**}{*/},
numbers=left,
numberstyle=\tiny\color{black},
stepnumber=1,
numbersep=3pt,
tabsize=8,                  % Groesse von Tabs
extendedchars=true,         %
breaklines=true,            % Zeilen werden Umgebrochen
% keywordstyle=\color{red},
frame=b,         
%        keywordstyle=[1]\textbf,    % Stil der Keywords
%        keywordstyle=[2]\textbf,    %
%        keywordstyle=[3]\textbf,    %
%        keywordstyle=[4]\textbf,   \sqrt{\sqrt{}} %
% stringstyle=\color{white}\ttfamily, % Farbe der String
showspaces=false,           % Leerzeichen anzeigen ?
showtabs=false,             % Tabs anzeigen ?
xleftmargin=10pt,
framexleftmargin=17pt,
framexrightmargin=5pt,
framexbottommargin=4pt,
% backgroundcolor=\color{lightgray},
showstringspaces=false      % Leerzeichen in Strings anzeigen ?        
}

\usepackage{caption}
\DeclareCaptionFont{white}{\color{white}}
\DeclareCaptionFormat{listing}{\colorbox[cmyk]{0.43, 0.35, 0.35,0.10}{\parbox{\textwidth}{\hspace{15pt}#1#2#3}}}
\captionsetup[lstlisting]{format=listing,labelfont=white,textfont=white, singlelinecheck=false, margin=0pt, font={bf,footnotesize}}

%----------------------------
\usepackage{anysize}
\marginsize{4cm}{3cm}{2cm}{2cm}
%----------------------------
\newcommand{\toref}[1]{\textbf{[REFERENCE: #1]}}
\newcommand{\tocomplete}[1]{\textbf{[Complete: #1 ]}}
\newcommand{\todelete}[1]{\textbf{[Delete this: \textcolor{cyan}{#1}]}}
\newcommand{\tocheck}[1]{\textbf{[Check this!: #1 ]}}
\newcommand{\todo}[1]{\textbf{[To Do:{#1}]}}
\newcommand{\edited}[1]{\textcolor{red}{#1}}
\newcommand{\comment}[1]{\textcolor{blue}{#1}}
\newcommand{\newstuff}[1]{\textcolor{green}{#1}}
\newcommand{\mrk}[0]{}

%----------------------------
% espacio entre parrafos
\setlength{\parskip}{5mm}
% sangria
\setlength{\parindent}{5mm}

\linespread{1.1}%
\selectfont	
% correct bad hyphenation here
\hyphenation{ in-te-rac-ci�n conexio-nes }

%-----------------------------
\newcommand{\sm}[0]{{\em SM}}
\newcommand{\ie}[0]{{\em i.e.,}}
\newcommand{\eg}[0]{{\em e.g.}}
\newcommand{\etal}[0]{{\em et al.}}
\newcommand{\etc}[0]{{\em etc.}}

\newcommand{\cn}[0]{{\textit{concern}}}
\newcommand{\cns}[0]{{\textit{concerns}}}
\newcommand{\ccc}[0]{{\textit{crosscutting concerns}}}


\newcommand{\game}[0]{{\emph{Game}}}
\newcommand{\meters}[0]{{\emph{Meters}}}
\newcommand{\recall}[0]{{\emph{Game Recall}}}
\newcommand{\error}[0]{{\emph{Error Condition}}}
\newcommand{\errors}[0]{{\emph{Errors Conditions}}}
\newcommand{\comm}[0]{{\emph{Communication Protocol}}}
\newcommand{\comms}[0]{{\emph{Communication Protocols}}}
\newcommand{\demo}[0]{{\emph{Demo}}}
\newcommand{\resump}[0]{{\emph{Program Resumption}}}
\begin{document}
\fi
 
\chapter{Conclusiones y Trabajos Futuros}

En este trabajo se han estudiado interacciones entre aspectos en un dominio de la industria, como son las \textit{Slot Machines}.\newline
Se identificaron varios \ccc~ funcionales mediante el estudio de los \textit{concerns} m�s representativos en el dominio, 
tomando como punto de referencia los ya estudiados en la etapa de an�lisis de requerimientos \cite{zambranoAl:sac10}. 
% 
% Es escencial destacar que se implement� el \textit{software} de una \sm~ mediante el uso de \textit{Aspect Oriented Programming}, mecanismo efectivo 
% y aceptado para tratar la separaci�n de \textit{crosscutting concerns} \cite{Kiczales97}.

Sobre los \ccc~ funcionales y no funcionales del dominio, se identificaron varias interacciones. Las mismas se categorizaron 
seg�n la clasificaci�n de Sanen~\etal~\cite{Sanen06} en \textit{Mutex, Conflict, Dependency} y \textit{Reinforcement}.\newline
Se seleccion� una interacci�n para cada una de estas categor�as, para las cuales se estudi� un mecanismo que permite su tratamiento, 
realizando una implementaci�n concreta del mismo sobre el \textit{software} desarrollado.

Las cuatro interacciones fueron implementadas de forma tal que la \sm~ se comporte de la manera deseada. De cada uno de los mecanismos se puede destacar:
\begin{itemize}
\item \textit{Mutex}: se desarroll� un mecanismo modular que pudo ser generalizado y cuenta con la ventaja de ser independiente
del dominio subyacente. Una desventaja del mecanismo es que puede verse afectado por el \textit{fragil pointcut problem} \cite{ecoop-aaos03, KoppenStoerzer2004.eiwas}.

\item \textit{Reinforcement}: se implement\'o un mecanismo \textit{ad-hoc} mediante el uso de \textit{inter-type declarations}. El mismo permite que el sistema escale
de forma correcta cuando una nueva \textit{error condition} necesita ser reportada o cuando un \sm~utiliza un nuevo protocolo.

\item \textit{Conflict}: en este caso tambi\'en se implement� un mecanismo \textit{ad-hoc}, sobre el cual se detect� un patr�n para tratar las interacciones.
El mismo es aplicable en situaciones donde logre identificarse un \textit{join point}, que de evitar su ejecuci\'on prevenga de una
interacci\'on de este tipo.
\clearpage
\item \textit{Dependency}: se concluy\'o que no hace falta un mecanismo para tratar esta interacci�n en el contexto de trabajo a nivel de implementaci�n.
\end{itemize}

Para cada uno de los mecanismos desarrollados, se analizaron ventajas y desventajas con respecto a 
la mantenibilidad, genericidad, escalabilidad y modularizaci�n.	

En un sistema desarrollado usando orientaci\'on a aspectos los aspectos 
interact\'uan necesariamente de varias maneras. 
La implementaci\'on  de las interacciones puede realizarse de manera 
manual y ajustada a cada caso, pero esta pr\'actica es propensa a errores y 
genera problemas de mantenibilidad.
Por lo tanto, es necesario contar con soporte para las interacciones, en lo 
posible a nivel de lenguaje.


Se propone como trabajo a futuro, aplicar  los mecanismos propuestos en otros 
dominios, analizando si los mismos permiten implementar otras 
interacciones  pertenecientes a las  categor\'ias presentadas en este trabajo.
De esta manera se puede comprobar si las generalizaciones propuestas 
son reusables.  Por otra parte, es necesario profundizar el estudio de 
los mecanismos planteados e implementar formas alternativas para tratar las 
interacciones, con el objetivo de eliminar algunas de las limitaciones de los 
mecanismos propuestos.

En este trabajo se ha logrado estudiar e implementar las interacciones entre aspectos, 
analizando los desaf\'ios que ellas presentan, lo cual constituye el 
n\'ucleo de la propuesta de esta tesina.
El desarrollo de la misma ha contribuido 
a afianzar mis conocimientos y adquirir otros nuevos. Adem�s, representa una 
experiencia muy valiosa, contribuyendo a mi formaci�n profesional y personal.



\ifx\all\undefined
\end{document}
\fi