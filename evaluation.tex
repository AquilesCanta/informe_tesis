\ifx\all\undefined
\documentclass[a4paper,11pt]{book}

%----------------------------
\usepackage[utf8]{inputenc}
\usepackage[spanish]{babel}
\usepackage[T1]{fontenc}
%----------------------------
\usepackage{graphicx}
\usepackage[citecolor=black, urlcolor=black, linkcolor=black, colorlinks=true, bookmarksopen=true]{hyperref}
% \usepackage[toc]{glossaries} %xindy
% \makeglossaries
%----------------------------
\usepackage{color}
\usepackage{pdfpages}
%----------------------------
\newcommand{\sm}[0]{{\em SM}}
\newcommand{\ie}[0]{{\em i.e.,}}
\newcommand{\eg}[0]{{\em e.g.}}
\newcommand{\etal}[0]{{\em et al.}}
\newcommand{\etc}[0]{{\em etc.}}
\newcommand{\sdt}[0]{\textbf{SDT}}
\newcommand{\pat}[0]{\textbf{PAT}}
\newcommand{\cat}[0]{\textbf{CAT}}
\newcommand{\pmt}[0]{\textbf{PMT}}
\newcommand{\pid}[0]{\textbf{PID}}
\newcommand{\pes}[0]{\textbf{PES}}
\newcommand{\nit}[0]{\textbf{NIT}}





\begin{document}
\fi
 
\chapter{Evaluaci�n}

Este cap�tulo expone las pruebas realizadas sobre el sistema desarrollado y eval�a su comportamiento frente a distintas situaciones. El objetivo es encontrar limitaciones y mejores que realiza el dise�o sobre el formato com�n.

\section{Prueba con transport streams}

\begin{itemize}
	\item Hay que ver qu� pasa con el transporte de referencia
	\item Que pasa con uno con un mont�n de servicios
	\item Qu� pasa cuando se deja de emitir los flujos elementales a alguno
	\item 
\end{itemize}

\section{Lista de servicios extendida}

Ac� hay que hablar de la cantidad de servicios que pueden se�alizarse en un ts a diferencia de uno com�n.

Limitaci�n por ancho de banda para pmt, limitaci�n por tama�o de sdt. Limitaci�n por tama�o de pat. Limitaciones extras.

\section{Flujos elementales m�ltiples}

Hasta ahora es imposible incorporar distintos flujos elementales para un mismo servicio. Pero tampoco tendr�a sentido. Lo ideal ser�a mandar cada flujo de transporte en un grupo multicast distinto y que se logre algo similar a object based broadcasting.

\section{EPG y \emph{Closed Caption}}

Tienen que ir siempre por el flujo de transporte.

\section{Averiguar qu� onda con el scrambling y el acceso condicional} 

\section{Redes son ambientes inseguros}

\section{Es lo mismo SD o HD para el ts}
% Un transport stream que refiere a dos servicios diferidos.
% Dos transport streams que refieren a dos servicios diferentes. Uno cada uno.
% Un transport stream que refiere a muchos servicios. Digamos 10.
% Un transport stream que refiere a 2 servicios. Uno de los cuales no tiene emisi�n.
% Closed caption
% C�mo se comportar�a el acceso condicional con el esquema extendido
% Problemas de seguridad? Si alguien se pone a emitir algo en alg�n momento
% EIT deber�a funcionar
% Todo el resto de las cosas deber�an funcionar. Ver el tema del bitrate puede ser m�s complicado.
% Con este esquema es imposible usar varios audios.
% Tambi�n es imposible mandar video por multicast y audio por ts.
% Todos los flujos elementales se deben enviar por el mismo medio.
% Est� bueno que no le interesa al ts si el servicio es HD o SD. Inclusive podr�a cambiar, supongo yo.
% Que pasa si dos empiezan a emitir al mismo tiempo?


\ifx\all\undefined
\end{document}
\fi